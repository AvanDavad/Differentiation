\documentclass{article}
\usepackage[parfill]{parskip}
\usepackage[utf8]{inputenc}
\usepackage{amsmath}
\usepackage{amsthm} % for proofs
\usepackage{amssymb}
\usepackage{graphicx}
\usepackage{bm}
\usepackage[toc,page]{appendix}
\graphicspath{ {./images} }

\title{Differentiation}
\author{Dávid Iván}

\begin{document}

\maketitle

\tableofcontents

\newpage

\section{introduction}

When differentiating an arbitrary-shaped (scalar, vector, matrix or higher rank tensor) function $f$ w.r.t. a scalar (e.t. $t$, the time), the result ($\partial f / \partial t$, the Jacobian) has the same shape as the function.

When differentiating a scalar function w.r.t. $x$ (scalar, vector or matrix), the result has the same shape as $x^T$, i.e., the transpose of $x$.

Combining the 2 above, let's say for example that $f(x)$ is an n-vector, and $x$ is an m-vector. Then $\partial f(x) / \partial x$ is an n-by-m matrix. This can be generalized with tensors.

\section{scalar functions}

\subsection{differentiation w.r.t. a vector}

\paragraph{1.} Let $\mathbf{a} \in \mathbb{R}^n$ be a constant vector, $\mathbf{x} \in \mathbb{R}^n$. Then

\begin{equation} \label{eq:da_dx}
    \frac{d}{d\mathbf{x}} (\mathbf{x}^{T}\mathbf{a}) = \frac{d}{d\mathbf{x}} (\mathbf{a}^{T}\mathbf{x}) = \mathbf{a^T}
\end{equation}

\paragraph{2.} Let $\mathbf{A} \in \mathbb{R}^{nxn}$ be a constant matrix, $\mathbf{x} \in \mathbb{R}^n$ Then

\begin{equation} \label{eq:dA_dx}
    \frac{d}{d\mathbf{x}} (\mathbf{x}^{T}\mathbf{A}\mathbf{x}) = \mathbf{x}^T (\mathbf{A}^T + \mathbf{A})
\end{equation}

We can derive this as follows:

\[
\begin{split}
    \frac{d}{d\mathbf{x}} (\mathbf{x}^{T}\mathbf{A}\mathbf{x}) =& \frac{d}{d\mathbf{y}} (\mathbf{y}^{T}\mathbf{A}\mathbf{x}) + \frac{d}{d\mathbf{y}} (\mathbf{x}^{T}\mathbf{A}\mathbf{y})\\
    =& \frac{d}{d\mathbf{y}} (\mathbf{x}^{T}\mathbf{A}^T\mathbf{y}) + \frac{d}{d\mathbf{y}} (\mathbf{x}^T\mathbf{A}\mathbf{y})\\
    =& \mathbf{x}^T\mathbf{A}^T + \mathbf{x}^T\mathbf{A}\\
    =& \mathbf{x}^T (\mathbf{A}^T + \mathbf{A})
\end{split}
\]


\subsection{differentiation w.r.t. a matrix}

\paragraph{1.} Let $\mathbf{A} \in \mathbb{R}^{m\times n}$, $\mathbf{X} \in \mathbb{R}^{m\times n}$. Then

\begin{equation}
    \frac{d}{d\mathbf{X}}\text{Tr}(\mathbf{A^TX}) = \frac{d}{d\mathbf{X}}\text{Tr}(\mathbf{X^TA}) = \mathbf{A}
\end{equation}


\paragraph{2.} Let $\mathbf{A} \in \mathbb{R}^{m\times m}$, $\mathbf{X} \in \mathbb{R}^{m\times n}$. Then

\begin{equation}
    \frac{d}{d\mathbf{X}} \text{Tr}(\mathbf{X}^T \mathbf{A} \mathbf{X}) = (\mathbf{A} + \mathbf{A}^T) \mathbf{X}
\end{equation}

We can derive it as follows:

\[
\begin{split}
    \frac{d}{d\mathbf{X}} \text{Tr}(\mathbf{X}^T \mathbf{A} \mathbf{X}) =& \frac{d}{d\mathbf{Y}} \text{Tr}(\mathbf{Y}^T \mathbf{A} \mathbf{X}) + \frac{d}{d\mathbf{Y}} \text{Tr}(\mathbf{X}^T \mathbf{A} \mathbf{Y})\\
    =& \mathbf{X}^T \mathbf{A}^T + \mathbf{X}^T \mathbf{A}\\
    =& \mathbf{X}^T (\mathbf{A}^T + \mathbf{A})
\end{split}
\]

\paragraph{Example.} Consider now this example.

\[
f(\mathbf{X}) = \text{Tr}(\mathbf{X}^T \mathbf{A} \mathbf{X} \mathbf{B} \mathbf{X}^T \mathbf{C})
\]

where $\mathbf{X} \in \mathbb{R}^{n\times m}$, $\mathbf{A} \in \mathbb{R}^{n\times n}$, $\mathbf{B} \in \mathbb{R}^{m\times m}$, $\mathbf{C} \in \mathbb{R}^{n\times m}$.

\[
\begin{split}
    \frac{d}{d\mathbf{X}} f(\mathbf{X}) =& \frac{d}{d\mathbf{Y}} \text{Tr}(\mathbf{Y}^T \mathbf{A} \mathbf{X} \mathbf{B} \mathbf{X}^T \mathbf{C})\\
    +& \frac{d}{d\mathbf{Y}} \text{Tr}(\mathbf{X}^T \mathbf{A} \mathbf{Y} \mathbf{B} \mathbf{X}^T \mathbf{C})\\
    +& \frac{d}{d\mathbf{Y}} \text{Tr}(\mathbf{X}^T \mathbf{A} \mathbf{X} \mathbf{B} \mathbf{Y}^T \mathbf{C})
\end{split}
\]

Calculating these:

\[
\begin{split}
    \frac{d}{d\mathbf{Y}} \text{Tr}(\mathbf{Y}^T \mathbf{A} \mathbf{X} \mathbf{B} \mathbf{X}^T \mathbf{C}) = (\mathbf{A} \mathbf{X} \mathbf{B} \mathbf{X}^T \mathbf{C})^T = \mathbf{C}^T \mathbf{X} \mathbf{B}^T \mathbf{X}^T \mathbf{A}^T
\end{split}
\]

\[
\begin{split}
    \frac{d}{d\mathbf{Y}} \text{Tr}(\mathbf{X}^T \mathbf{A} \mathbf{Y} \mathbf{B} \mathbf{X}^T \mathbf{C}) =& \frac{d}{d\mathbf{Y}} \text{Tr}(\mathbf{Y} \mathbf{B} \mathbf{X}^T \mathbf{C} \mathbf{X}^T \mathbf{A})\\
    =& \mathbf{B} \mathbf{X}^T \mathbf{C} \mathbf{X}^T \mathbf{A}\\
\end{split}
\]

\[
\begin{split}
   \frac{d}{d\mathbf{Y}} \text{Tr}(\mathbf{X}^T \mathbf{A} \mathbf{X} \mathbf{B} \mathbf{Y}^T \mathbf{C}) &= \frac{d}{d\mathbf{Y}} \text{Tr}(\mathbf{Y}^T \mathbf{C} \mathbf{X}^T \mathbf{A} \mathbf{X} \mathbf{B})\\
   &= (\mathbf{C} \mathbf{X}^T \mathbf{A} \mathbf{X} \mathbf{B})^T = \mathbf{B}^T \mathbf{X}^T \mathbf{A}^T \mathbf{X} \mathbf{C}^T
\end{split}
\]

So the result is:

\[
\frac{d}{d\mathbf{X}} \text{Tr}(\mathbf{X}^T \mathbf{A} \mathbf{X} \mathbf{B} \mathbf{X}^T \mathbf{C}) = \mathbf{C}^T \mathbf{X} \mathbf{B}^T \mathbf{X}^T \mathbf{A}^T + \mathbf{B} \mathbf{X}^T \mathbf{C} \mathbf{X}^T \mathbf{A} + \mathbf{B}^T \mathbf{X}^T \mathbf{A}^T \mathbf{X} \mathbf{C}^T
\]


\end{document}